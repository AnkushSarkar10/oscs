\documentclass[letterpaper, 8pt]{extarticle}
\usepackage{amssymb,amsmath,amsthm,amsfonts}
\usepackage{multicol,multirow}
\usepackage{calc}
\usepackage{ifthen}
\usepackage[landscape]{geometry}
\usepackage[colorlinks=true,citecolor=blue,linkcolor=blue]{hyperref}
\usepackage{booktabs}
\usepackage{ulem}
\usepackage{enumitem}
\usepackage{tabulary}
\usepackage{graphicx}
\usepackage{siunitx}
\usepackage{tikz}
\usepackage{derivative}
\usepackage{svg}
\usepackage{listings}


\ifthenelse{\lengthtest { \paperwidth = 11in}}
    { \geometry{top=.25in,left=.25in,right=.25in,bottom=.25in} }
	{\ifthenelse{ \lengthtest{ \paperwidth = 297mm}}
		{\geometry{top=1cm,left=1cm,right=1cm,bottom=1cm} }
		{\geometry{top=1cm,left=1cm,right=1cm,bottom=1cm} }
	}

\newenvironment{Figure}
  {\par\medskip\noindent\minipage}
  {\endminipage\par\medskip}

\pagestyle{empty}
\makeatletter
\renewcommand{\section}{\@startsection{section}{1}{0mm}%
                                {-1ex plus -.5ex minus -.2ex}%
                                {0.5ex plus .2ex}%x
                                {\normalfont\normalsize\bfseries}}
\renewcommand{\subsection}{\@startsection{subsection}{2}{0mm}%
                                {-1explus -.5ex minus -.2ex}%
                                {0.5ex plus .2ex}%
                                {\normalfont\small\bfseries}}
\renewcommand{\subsubsection}{\@startsection{subsubsection}{3}{0mm}%
                                {-1ex plus -.5ex minus -.2ex}%
                                {1ex plus .2ex}%
                                {\normalfont\tiny\bfseries}}
\makeatother
\setcounter{secnumdepth}{0}
\setlength{\parindent}{0pt}
\setlength{\parskip}{0pt plus 0.5ex}
% -----------------------------------------------------------------------
% \tymin=37pt
% \tymax=\maxdimen

% Custom siunitx defs
\DeclareSIUnit\noop{\relax}

\NewDocumentCommand\prefixvalue{m}{%
\qty[prefix-mode=extract-exponent,print-unity-mantissa=false]{1}{#1\noop}
}

% Shorthand definitions
\newcommand{\To}{\Rightarrow}

% condense itemize & enumerate
\let\olditemize=\itemize \let\endolditemize=\enditemize \renewenvironment{itemize}{\olditemize \itemsep0em}{\endolditemize}
\let\oldenumerate=\enumerate \let\endoldenumerate=\endenumerate \renewenvironment{enumerate}{\oldenumerate \itemsep0em}{\endoldenumerate}

\title{2C03}

\begin{document}

\raggedright
\tiny

\begin{center}
    {\textbf{2C03}} \\
\end{center}
\begin{multicols*}{5}
    \setlength{\premulticols}{1pt}
    \setlength{\postmulticols}{1pt}
    \setlength{\multicolsep}{1pt}
    \setlength{\columnsep}{2pt}

    \section{Basic DSes}
    Array vs LL

    \textbf{Arrays:}
    Static data sets, ones where fast index-based access is needed,
    Slow expansion, insertion, deletion.
    \textbf{LLs:}
    Dynamic data sets,
    insertion \& deletion more important than random access.
    % TODO: Probably don't need LLs,
    % pretty easy to reason through
    \subsection{Linked Lists}
    \begin{lstlisting}[language=Java, breaklines=true, postbreak=\mbox{\textcolor{red}{$\hookrightarrow$}\space}]
class Node {
    Item item;
    Node next;
    Node prev; // for DLL
}
    \end{lstlisting}
    \subsubsection{Operations}
    \begin{tabular}[!ht]{@{}ll@{}} \toprule
        Search           & O(N) \\
        Prepend / Append & O(1) \\
        Delete           & O(N) \\
        \bottomrule
    \end{tabular}

    \section{Runtime Analysis}
    \begin{tabular}[!ht]{@{}lc@{}} \toprule
        constant     & 1          \\
        logarithmic  & $\log N$   \\
        linear       & $N$        \\
        linearithmic & $N \log N$ \\
        quadratic    & $N^2$      \\
        cubic        & $N^3$      \\
        exponential  & $2^N$      \\
        \bottomrule
    \end{tabular}

    \subsection{Proving}
    For any pair of constants, $c, n_0$, consider $n = X[c \cdot n_0]$,
    where $X$ is another arbitrarily chosen constant.

    If a pair can be found that satisfies these constants, then list them.
    If a pair cannot be found, prove that there exists a contradiction.

    \subsection{Big-O}
    Upper limit, function must be $\leq O(n)$

    $\lim_{n \to \infty} \frac{f(n)}{g(n)} \neq \infty$

    \subsection{Big-$\Omega$}
    Lower limit, function must be $\geq \Omega(n)$

    $\lim_{n \to \infty} \frac{f(n)}{g(n)} \neq 0$

    \subsection{Big-$\Theta$}
    Must satisfy both Big-O and Big-$\Omega$.

    $\lim_{n \to \infty} \frac{f(n)}{g(n)} \not\in \{0, \infty\}$

    \subsection{Recurrance Equations}
    % TODO: fixme

    \section{Union-Find}
    Key learnings:
    \begin{itemize}
        \item Generally, either optimize for adding data into Data Structure,
              or optimize for processing the Data Structure for desired results.

        \item
    \end{itemize}

    \section{Bad Sorts}
    \subsection{Selection Sort}
    Build up a sorted section of array, by
    \begin{enumerate}
        \item Find min element
        \item Swap min \& first element
        \item Examine array, skipping first element.
    \end{enumerate}
    $\Theta(N^2)$
    \subsection{Insertion Sort}
    Start from 2 elements, build up sorted subarray,
    "inserting" a new element each iteration by swapping
    until it is in the right position.
    $\Omega(N), O(n^2)$
    \subsection{Shellsort}
    Pick every $n$ elements and put them in a subarray,
    sort them, and put them back.
    (If you only use indicies you don't need to make a new array.)

    \section{Good Sorts}
    \subsection{Mergesort}
    Divide array in half recursively, until it is down to 1 element.
    Merge array together,
    making sure each side is sorted as it is merged together.

    \begin{lstlisting}[language=Java, breaklines=true, postbreak=\mbox{\textcolor{red}{$\hookrightarrow$}\space}]
private static void merge(Comparable[] a, Comparable[] aux, int lo, int mid, int hi) {
    // copy
    for (int k = lo; k <= hi; k++)
        aux[k] = a[k];
    // merge
    int i = lo, j = mid + 1;
    for (knt i = lo; k <= hi; k++) {
        if (i > mid) 
            a[k] = aux[j++];
        else if
            (j > hi) a[k] = aux[i++];
        else if
            (less(aux[j], aux[i])) a[k] = aux[j++];
        else
            a[k] = aux(i++);
    }
}
    \end{lstlisting}

    \subsection{Quicksort}

    \section{Priority Queues}
    Supports insertion and removing/popping the priority (largest or smallest) item.

    Implementations:
    \begin{list}{}{}
        \item Sorted Array - O(n) insert, O(1) pop
        \item Unsorted Array - O(1) insert, O(n) pop
        \item Binary Heap - O(log n) insert and sort
    \end{list}

    \section{Binary Heaps \& Heapsort}
    A max binary heap is a complete binary tree where the keys are in the nodes and each parent's key $\geq$ each child's key. This requirement is called the max heap property.
    \\
    Binary Heaps:
    \begin{itemize}
        \item Can be represented as array or tree/nodes.
        \item Insertion: We insert at the end of the array, then "swim up" the value.
        \item Swimming up - exchange a given node with it's parent until the binary max property is fulfilled.
        \item Popping - swap the first node (the max) with the last node, remove it, then "sink" the first node.
        \item Sinking - exchange a given node with the max of it's children until the binary max property is fulfilled.
    \end{itemize}

\end{multicols*}

\end{document}
