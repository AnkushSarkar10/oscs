\documentclass[letterpaper, 8pt]{extarticle}
\usepackage{amssymb,amsmath,amsthm,amsfonts}
\usepackage{multicol,multirow}
\usepackage{calc}
\usepackage{ifthen}
\usepackage[landscape]{geometry}
\usepackage[colorlinks=true,citecolor=blue,linkcolor=blue]{hyperref}
\usepackage{booktabs}
\usepackage{ulem}
\usepackage{enumitem}
\usepackage{tabulary}
\usepackage{graphicx}
\usepackage{siunitx}
\usepackage{tikz}
\usepackage{derivative}


\ifthenelse{\lengthtest { \paperwidth = 11in}}
    { \geometry{top=.25in,left=.25in,right=.25in,bottom=.25in} }
	{\ifthenelse{ \lengthtest{ \paperwidth = 297mm}}
		{\geometry{top=1cm,left=1cm,right=1cm,bottom=1cm} }
		{\geometry{top=1cm,left=1cm,right=1cm,bottom=1cm} }
	}

\newenvironment{Figure}
  {\par\medskip\noindent\minipage}
  {\endminipage\par\medskip}

\pagestyle{empty}
\makeatletter
\renewcommand{\section}{\@startsection{section}{1}{0mm}%
                                {-1ex plus -.5ex minus -.2ex}%
                                {0.5ex plus .2ex}%x
                                {\normalfont\normalsize\bfseries}}
\renewcommand{\subsection}{\@startsection{subsection}{2}{0mm}%
                                {-1explus -.5ex minus -.2ex}%
                                {0.5ex plus .2ex}%
                                {\normalfont\small\bfseries}}
\renewcommand{\subsubsection}{\@startsection{subsubsection}{3}{0mm}%
                                {-1ex plus -.5ex minus -.2ex}%
                                {1ex plus .2ex}%
                                {\normalfont\tiny\bfseries}}
\makeatother
\setcounter{secnumdepth}{0}
\setlength{\parindent}{0pt}
\setlength{\parskip}{0pt plus 0.5ex}
% -----------------------------------------------------------------------
% \tymin=37pt
% \tymax=\maxdimen

% Custom siunitx defs
\DeclareSIUnit\noop{\relax}

\NewDocumentCommand\prefixvalue{m}{%
\qty[prefix-mode=extract-exponent,print-unity-mantissa=false]{1}{#1\noop}
}

% Shorthand definitions
\newcommand{\To}{\Rightarrow}

\title{2LC3}

\begin{document}

\raggedright
\tiny

\begin{center}
    {\textbf{2LC3 - Jason Huang}} \\
\end{center}
\begin{multicols*}{3}
    \setlength{\premulticols}{1pt}
    \setlength{\postmulticols}{1pt}
    \setlength{\multicolsep}{1pt}
    \setlength{\columnsep}{2pt}

    \section{Theorems}
    \tiny
    \subsection{Equivalence and true}
    \begin{tabular}{@{}lll@{}}
        (3.1) & Axiom, Associativity of $\equiv$ & $((p \equiv q) \equiv r) \equiv (p \equiv (q \equiv r))$ \\
        (3.2) & Axiom, Symmetry of $\equiv$      & $p \equiv q \equiv q \equiv p$                           \\
        (3.3) & Axiom, Identity of $\equiv$      & $\textit{true} \equiv q \equiv q$                        \\
        (3.4) &                                  & \textit{true}                                            \\
        (3.5) & Reflexivity of $\equiv$          & $p \equiv p$                                             \\
    \end{tabular}

    \subsection{Negation, inequivalence, and false}
    \begin{tabular}{@{}lll@{}}
        (3.8)  & Axiom, Definition of \textit{false}           & $\textit{false} \equiv \neg \textit{true}$                                  \\
        (3.9)  & Axiom, Distributivity of $\neg$ over $\equiv$ & $\neg (p \equiv q) \ \equiv \ \neg \equiv q$                                \\
        (3.10) & Axiom, Definition of $\not\equiv$             & $(p \not\equiv q) \ \equiv \ \neg(p \equiv q)$                              \\
        (3.11) &                                               & $\neg p \equiv q \equiv p \equiv \neg q$                                    \\
        (3.12) & Double negation                               & $\neg \neg p \equiv p$                                                      \\
        (3.13) & Negation of \textit{false}                    & $\neg \textit{false} \equiv \textit{true}$                                  \\
        (3.14) &                                               & $(p \not\equiv q) \ \equiv \ \neg p \equiv q$                               \\
        (3.15) &                                               & $(\neg p \equiv p \equiv false)$                                            \\
        (3.16) & Symmetry of $\not\equiv$                      & $(p \not\equiv q) \ \equiv \ (q \not\equiv p)$                              \\
        (3.17) & Associativity of $\not\equiv$                 & $((p \not\equiv q) \not\equiv r) \ \equiv \ (p \not\equiv(q \not\equiv r))$ \\
        (3.18) & Mutual Associativity                          & $((p \not\equiv q) \equiv r) \ \equiv \ (p \not\equiv (q \equiv r))$        \\
        (3.19) & Mutual interchangeability                     & $p \not\equiv q \equiv r \ \equiv \ p \equiv q \not\equiv r$                \\
    \end{tabular}

    \subsection{Disjunction}
    \begin{tabular}{@{}lll@{}}
        (3.24) & Axiom, Symmetry of $\lor$                     & $p \lor q \equiv q \lor p$                            \\
        (3.25) & Axiom, Associativity of $\lor$                & $(p \lor q) \lor r \equiv p \lor (q \lor r)$          \\
        (3.26) & Axiom, Idempotency of $\lor$                  & $p \lor p \equiv p$                                   \\
        (3.27) & Axiom, Distributivity of $\lor$ over $\equiv$ & $p \lor(q \equiv r) \equiv p \lor q \equiv p \lor r$  \\
        (3.28) & Axiom, Excluded Middle                        & $p \lor \neg p$                                       \\
        (3.29) & Zero of $\lor$                                & $p \lor \textit{true} \equiv \textit{true}$           \\
        (3.30) & Identity of $\lor$                            & $p \lor false \equiv p$                               \\
        (3.31) & Distributivity of $\lor$ over $\lor$          & $p \lor (q \lor r) \equiv (p \lor q) \lor (p \lor r)$ \\
        (3.32) &                                               & $p \lor q \equiv p \lor \neg q \equiv p$              \\
    \end{tabular}

    \subsection{Conjunction}
    \begin{tabular}{@{}lll@{}}
        (3.35) & Axiom, Golden rule                     & $p \land q \equiv p \equiv q \equiv p \lor q$                                  \\
        (3.36) & Symmetry of $\land$                    & $p \land q \equiv q \land p$                                                   \\
        (3.37) & Associativity of $\land$               & $(p \land q) \land r \equiv p \land(q \land r)$                                \\
        (3.38) & Idempotency of $\land$                 & $p \land p \equiv p$                                                           \\
        (3.39) & Identity of $\land$                    & $p \land \textit{true} \equiv p$                                               \\
        (3.40) & Zero of $\land$                        & $p \land \textit{false} \equiv \textit{false} $                                \\
        (3.41) & Distributivity of $\land$ over $\land$ & $p \land (q \land r) \equiv (p \land q) \land (p \land r)$                     \\
        (3.42) & Contradiction                          & $p \land \neg p \equiv \textit{false}$                                         \\
        (3.43) & Absorption                             & (a) $p \land (p \lor q) \equiv p$                                              \\
               &                                        & (b) $p \lor (p \land q) \equiv p$                                              \\
        (3.44) & Absorption                             & (a) $p \land (\neg p \lor q) \equiv p \land q$                                 \\
               &                                        & (b) $p \lor (\neg p \land q) \equiv p \lor q$                                  \\
        (3.45) & Distributivity of $\lor$ over $\land$  & $p \lor (q \land r) \equiv (p \lor q) \land (p \lor r)$                        \\
        (3.46) & Distributivity of $\land$ over $\lor$  & $p \land (q \lor r) \equiv (p \land q) \lor (p \land r)$                       \\
        (3.47) & De Morgan                              & (a) $\neg (p \land q) \equiv \neg p \lor \neg q)$                              \\
               &                                        & (b) $\neg (p \lor q) \equiv \neg p \land \neg q)$                              \\
        (3.48) &                                        & $p \land q \equiv p \land \neg q \equiv \neg p$                                \\
        (3.49) &                                        & $p \land (q \equiv r) \equiv p \land q \equiv p \land r \equiv p$              \\
        (3.50) &                                        & $p \land (q \equiv p) \equiv p \land q$                                        \\
        (3.51) & Replacement                            & $(p \equiv q) \land (r \equiv p) \ \equiv \ (p \equiv q) \lor (r \equiv q)$    \\
        (3.52) & Definition of $\equiv$                 & $p \equiv q \equiv (p \land q) \lor (\neg p \land \neg q)$                     \\
        (3.53) & Exclusive or                           & $p \not\equiv q \equiv (\neg p \land q) \lor (p \land \neg q)$                 \\
        (3.55) &                                        & $(p \land q) \land r \equiv \ p \equiv q \equiv r$                             \\
               &                                        & \quad $\equiv p \lor q \equiv q \lor r \equiv r \lor p \equiv p \lor q \lor r$ \\
    \end{tabular}

    \subsection{Implication}
    \begin{tabular}{@{}lll@{}}
        (3.57) & Axiom, Definition of Implication      & $p \To q \equiv p \lor q \equiv q$                      \\
        (3.58) & Axiom, Consequence                    & $p \Leftarrow q \equiv q \To p$                         \\
        (3.59) & Definition of implication             & $p \To q \equiv \neg p \lor q$                          \\
        (3.60) & Definition of implication             & $p \To q \equiv p \land q \equiv p$                     \\
        (3.61) & Contrapositive                        & $p \To q \equiv \neg q \To \neg p$                      \\
        (3.62) &                                       & $p \To (q \equiv r) \equiv p \land q \equiv p \land r$  \\
        (3.63) & Distributivity of $\To$ over $\equiv$ & $p \To (q \equiv r) \equiv p \To q \equiv p \To r$      \\
        (3.64) &                                       & $p \To (q \To r) \equiv (p \To q) \To (p \To r)$        \\
        (3.65) & Shunting                              & $p \land q \To r \equiv p \To (q \To r)$                \\
        (3.66) &                                       & $p \land (p \To q) \equiv p \land q$                    \\
        (3.67) &                                       & $p \land (q \To p) \equiv p$                            \\
        (3.68) &                                       & $p \lor (p \To q) \equiv \textit{true}$                 \\
        (3.69) &                                       & $p \lor (q \To p) \equiv q \To p$                       \\
        (3.70) &                                       & $p \lor q \To p \land q \equiv p \equiv q$              \\
        (3.71) & Reflexivity of $\To$                  & $p \To p \equiv \textit{true}$                          \\
        (3.72) & Right zero of $\To$                   & $p \To \textit{true} \equiv \textit{true}$              \\
        (3.73) & Left identity of $\To$                & $\textit{true} \To p \equiv p$                          \\
        (3.74) &                                       & $p \To \textit{false} \equiv \neg p$                    \\
        (3.75) &                                       & $\textit{false} \To p \equiv true$                      \\
        (3.76) & Weakening/strengthening               & (a) $p \To p \lor q$                                    \\
               &                                       & (b) $p \land q \To p$                                   \\
               &                                       & (c) $p \land q \To p \lor q$                            \\
               &                                       & (d) $p \lor (q \land r) \To p \lor q$                   \\
               &                                       & (e) $p \land q \To p \land (q \lor r)$                  \\
        (3.77) & Modus ponens                          & $p \land (p \To q) \To q$                               \\
        (3.78) &                                       & $(p \To r) \land (q \To r) \ \equiv \ (p \lor q \To r)$ \\
        (3.79) &                                       & $(p \To r) \land (\neg p \To r) \ \equiv \ r$           \\
        (3.80) & Mutual implication                    & $(p \To q) \land (q \To p) \equiv (p \equiv q)$         \\
        (3.81) & Antisymmetry                          & $(p \To q) \land (q \To p) \To (p \equiv q)$            \\
        (3.82) & Transitivity                          & (a) $(p \To q) \land (q \To r) \To (p \To r)$           \\
               &                                       & (b) $(p \equiv q) \land (q \To r) \To (p \To r)$        \\
               &                                       & (c) $(p \To q) \land (q \equiv r) \To (p \To r)$        \\
    \end{tabular}

    \subsection{Leibniz as an axiom}
    \begin{tabulary}{\linewidth}{@{}llL@{}}
        (3.83) & Axiom, Leibniz            & $e = f \ \To \ E_e^z = E_f^z$                                                         \\
        (3.84) & Substitution              & (a) $(e = f) \land E_e^z \ \equiv \ (e = f) \land E_f^z$                              \\
               &                           & (b) $(e = f) \To E_e^z \ \equiv \ (e = f) \To E_f^z$                                  \\
               &                           & (c) $q \land (e = f) \To E_e^z \ \equiv \ q \land (e = f) \To E_f^z$                  \\
        (3.85) & Replace by \textit{true}  & (a) $p \To E_b^z \ \equiv \ p \To E_\textit{true}^z$                                  \\
               &                           & (b) $q \land p \To E_p^z \ \equiv \ q \land p \To E_\textit{true}^z$                  \\
        (3.86) & Replace by \textit{false} & (a) $E_p^z \To p \ \equiv \ E_\textit{false}^z \To p$                                 \\
               &                           & (b) $E_p^z \To p \lor q \ \equiv \ E_\textit{false}^z \To p \lor q$                   \\
        (3.87) & Replace by \textit{true}  & $p \land E_p^z \ \equiv \ p \land E_\textit{true}^z$                                  \\
        (3.88) & Replace by \textit{false} & $p \lor E_p^z \ \equiv \ p \lor E_\textit{false}^z$                                   \\
        (3.89) & Shannon                   & $E_p^z \ \equiv \ (p \land E_\textit{true}^z) \lor (\neg p \land E_\textit{false}^z)$ \\
        (4.1)  &                           & $p \To (q \To p)$                                                                     \\
        (4.2)  & Monotonicity of $\lor$    & $(p \To q) \To (p \lor r \To q \lor r)$                                               \\
        (4.3)  & Monotonicity of $\land$   & $(p \To q) \To (p \land r \To q \land r)$                                             \\
    \end{tabulary}

    \subsection{Proof techniques}
    \begin{tabulary}{\linewidth}{@{}llL@{}}
        (4.4)  & Deduction               & To prove $P \To Q$, assume $P$ and prove $Q$.                                  \\
        (4.5)  & Case analysis           & If $E_\textit{true}^z$, $E_\textit{false}^z$ are theorems, then so is $E_P^z$. \\
        (4.6)  & Case analysis           & $(p \lor q \lor r) \land (p \To s) \land (q \To s) \land(r \To s) \To \ s$     \\
        (4.7)  & Mutual implication      & To prove $P \equiv Q$, prove $P \To Q$ and $Q \To P$.                          \\
        (4.9)  & Proof by contradiction  & To prove $P$, prove $\neg P \To \textit{false}$.                               \\
        (4.12) & Proof by contrapositive & To prove $P \To Q$, prove $\neg Q \To \neg P$                                  \\
    \end{tabulary}

    \section{General Laws of Quantification}
    For symmetric and associative binary operator $\star $ with identity $u$.
    \begin{tabulary}{\linewidth}{@{}lL@{}}
        (8.13) & \textbf{Axiom, Empty range:} $(\star x | false : P) = u$                                                                                                                                                                                   \\
        (8.14) & \textbf{Axiom, One-point rule:} Provided $\neg occurs(`x\textrm', `E\textrm')$, $(\star x | x = E : P) = P[x := E]$                                                                                                                        \\
        (8.15) & \textbf{Axiom, Distributivity:} Provided each quantification is defined, $(\star x | R:P) \star  (\star x | R : Q) = (\star x | R : P \star  Q)$                                                                                           \\
        (8.16) & \textbf{Axiom, Range split:} Provided $R \land S \equiv false$ and each quantification is defined, $(\star x | R \lor S : P) = (\star x | R \land S : P) = (\star x | R : P) \star  (\star X | S : P)$                                     \\
        (8.17) & \textbf{Axiom, Range split:} Provided each quantification is defined, $(\star x | R \lor S : P) \star  (\star x | R \land S : P) = (\star x | R : P) \star  (\star x | S : P)$                                                             \\
        (8.18) & \textbf{Axiom, Range split for idempotent $\star$:} Prov. each quant. is defined, $(\star x | R \lor S : P) = (\star x | R : P) \star (\star x | S : P)$                                                                                   \\
        (8.19) & \textbf{Axiom, Interchange of dummies:} Provided each quantification is defined, $\neg occurs(`y\textrm', `R\textrm')$, and $\neg occurs(`x\textrm', `Q\textrm')$, $(\star x | R : (\star y | Q : P)) = (\star y | Q : (\star x | R : P))$ \\
        (8.20) & \textbf{Axiom, Nesting:} Provided $\neg occurs(`y\textrm', `R\textrm')$, $(\star x, y | R \land Q : P) = (\star x | R : (\star y | Q : P))$                                                                                                \\
        (8.21) & \textbf{Axiom, Dummy renaming:} Provided $\neg occurs(`y\textrm', `R, P\textrm')$, $(\star x | R : P) = (\star y | R[x := y] : P [x := y])$                                                                                                \\
        (8.22) & \textbf{Change of dummy:} Provided $\neg occurs(`y\textrm', `R, P\textrm')$, and $f$ has an inverse, $(\star x | R : P) = (\star y | R[x := f.y] : P[x := f.y])$                                                                           \\
        (8.23) & \textbf{Split off term:} $(\star i | 0 \leq i < n + 1 : P) = (\star i | 0 \leq i < n : P) \star P_n^i$                                                                                                                                     \\
    \end{tabulary}

    \section{Theorems of the Predicate Calculus}
    \subsection{Universal quantification}
    \begin{tabulary}{\linewidth}{@{}lLL@{}}
        (9.2)  & \textbf{Axiom, Trading:}                                  & $(\forall x | R : P) \equiv (\forall x |: R \To P)$                                                                                              \\
        (9.3)  & \textbf{Trading:}                                         & (a) $(\forall x | R : P) \equiv (\forall x |: \neg R \lor P)$                                                                                    \\
               &                                                           & (b) $(\forall x | R : P) \equiv (\forall x |: R \land P \equiv R)$                                                                               \\
               &                                                           & (c) $(\forall x | R : P) \equiv (\forall x |: R \lor P \equiv P)$                                                                                \\
        (9.4)  & \textbf{Trading:}                                         & (a) $(\forall x | Q \land R : P) \equiv (\forall x | Q : R \To P)$                                                                               \\
               &                                                           & (b) $(\forall x | Q \land R : P) \equiv (\forall x | Q : \neg R \lor P)$                                                                         \\
               &                                                           & (c) $(\forall x | Q \land R : P) \equiv (\forall x | Q : R \land P \equiv R)$                                                                    \\
               &                                                           & (d) $(\forall x | Q \land R : P) \equiv (\forall x | Q : R \lor P \equiv P)$                                                                     \\
        (9.5)  & \textbf{Axiom, Distributivity of $\lor $ over $\forall$:} & Prov. $\neg occurs(`x\textrm', `P\textrm')$, $P \lor (\forall x | R : Q) \equiv (\forall x | R : P \lor Q)$                                      \\
        (9.6)  &                                                           & Provided $\neg occurs(`x\textrm', `P\textrm')$, $(\forall x | R : P) \equiv P \lor (\forall x |: \neg R)$                                        \\
        (9.7)  & \textbf{Distributivity of $\land$ over $\forall$:}        & Provided $\neg occurs(`x\textrm', `P\textrm')$, $\neg(\forall x |: \neg R) \To ((\forall x | R : P \land Q) \equiv P \land (\forall x | R : Q))$ \\
        (9.8)  &                                                           & $(\forall x | R : true) \equiv true$                                                                                                             \\
        (9.9)  &                                                           & $(\forall x | R : P \equiv Q) \To ((\forall x | R : P) \equiv (\forall x | R : Q))$                                                              \\
        (9.10) & \textbf{Range weakening/strengthening:}                   & $(\forall x | Q \lor R : P) \To (\forall x | Q : P)$                                                                                             \\
        (9.11) & \textbf{Body weakening/strengthening:}                    & $(\forall x | R : P \land Q) \To (\forall x | R : P)$                                                                                            \\
        (9.12) & \textbf{Monotonicity of $\forall$:}                       & $(\forall x | R : Q \To P) \To ((\forall x | R : Q) \To (\forall x | R : P))$                                                                    \\
        (9.13) & \textbf{Instantiation:}                                   & $(\forall x |: P) \To P[x := e]$                                                                                                                 \\
        (9.16) &                                                           & $P$ is a theorem iff $(\forall x |: P)$ is a theorem.                                                                                            \\
    \end{tabulary}

    \subsection{Existential quantification}
    \begin{tabulary}{\linewidth}{@{}lLL@{}}
        (9.17) & \textbf{Axiom, Generalized De Morgan:}             & $(\exists x | R : P) \equiv \neg (\forall x | R : \neg P)$                                                                                                                 \\
        (9.18) & \textbf{Generalized De Morgan:}                    & (a) $\neg (\exists x | R : \neg P) \equiv (\forall x | R : P)$                                                                                                             \\
               &                                                    & (b) $\neg(\exists x | R : P) \equiv (\forall x | R : \neg P)$                                                                                                              \\
               &                                                    & (c) $(\exists x | R : \neg P) \equiv \neg(\forall x | R : P)$                                                                                                              \\
        (9.19) & \textbf{Trading:}                                  & $(\exists x | R : P) \equiv (\exists x |: R \land P)$                                                                                                                      \\
        (9.20) & \textbf{Trading:}                                  & $(\exists | Q \land R : P) \equiv (\exists x | Q : R \land P)$                                                                                                             \\
        (9.21) & \textbf{Distributivity of $\land$ over $\exists$:} & Provided $\neg occurs(`x\textrm', `P\textrm')$, $P \land (\exists x | R : Q) \equiv (\exists x | R : P \land Q)$                                                           \\
        (9.22) &                                                    & $(\exists x | R : false) \equiv false$                                                                                                                                     \\
        (9.23) & \textbf{Distributivity of $\lor$ over $\exists$:}  & Provided $\neg occurs(`x\textrm', `P\textrm')$, $(\equiv x |: R) \To ((\equiv x | R : P \lor Q) \equiv P \lor (\exists x | R : Q))$                                        \\
        (9.24) &                                                    & $(\exists x | R : false) \equiv false$                                                                                                                                     \\
        (9.25) & \textbf{Range weakening/strengthening:}            & $(\exists x | R : P) \To (\exists x | Q \lor R : P)$                                                                                                                       \\
        (9.26) & \textbf{Body weakening/strengthening:}             & $(\exists x | R : P) \To (\exists x | R : P \lor Q)$                                                                                                                       \\
        (9.27) & \textbf{Monotonicity of $\exists$:}                & $(\forall x | R : Q \To P) \To ((\exists x | R : Q) \To (\exists x | R : P ))$                                                                                             \\
        (9.28) & \textbf{$\exists$-Introduction:}                   & $P[x := E] \To (\exists x |: P)$                                                                                                                                           \\
        (9.29) & \textbf{Interchange of quantifications:}           & Provided $\neg occurs(`y\textrm', `R\textrm')$ and $\neg occurs(`x\textrm', `Q\textrm')$, $(\exists x | R: (\forall y | Q : P)) \To (\forall y | Q : (\exists x | R : P))$ \\
        (9.30) &                                                    & Provided $\neg occurs(`\hat{x}\textrm', `Q\textrm')$, $(\exists x | R : P) \To Q \text{ is a theorem iff } (R \land P)[x := \hat{x}] \To Q \text{ is a theorem}$           \\
    \end{tabulary}

    \section{LN1}
    Inference rule: $\dfrac{P_1,...,P_k}{C}$, where $P_i$ - premises or hypoth., C is concl.

    Inference rule asserts that if the premises are theorems, then the conclusion is a theorem.

    Inference rule Substitution: E-expression, v - list of variables, F - list of expressions. $\dfrac{E}{E[v:=F]}$.

    Laws: Reflexivity. $x=x$, Symmetry. $(x = y) = (y= x)$, Transitivity $\dfrac{X=Y, Y = Z}{X =Z}$, Leibnitz $\dfrac{X=Y}{E[z := X]=E[z := Y]}$

    A precondition of a statement is an assertion about the
    program variables in a state in which the statement may be
    executed. A postcondition is an assertion about the states in which it
    may terminate.

    Hoare Triple - a notation: $\{P\} S \{Q\}$
    Assignment := $\{R[x := E]\} x := E \{R\}$.

    \section{LN2}

    The \textit{dual} $P_D$ of a boolean expression $P$ is constructed by swapping:
    $true$ and $false$,
    $\land$ and $\lor$,
    $\equiv$ and $\not\equiv$,
    $\Rightarrow$ and $\nLeftarrow$,
    $\Leftarrow$ and $\nRightarrow$.

    \begin{center}
        \begin{tabulary}{\linewidth}{@{}LLL@{}}
            \toprule
            and, but                & becomes & $\land$   \\
            or                      & becomes & $\lor$    \\
            not                     & becomes & $\neg$    \\
            it is not the case that & becomes & $\neg$    \\
            if $p$ then $q$         & becomes & $p \To q$ \\
            means                   & becomes & $\equiv$  \\
            however                 & becomes & $\land$   \\
            ;                       & becomes & $\land$   \\
            \bottomrule
        \end{tabulary}
        \begin{tabulary}{\linewidth}{@{}C|CCCC@{}}
            \toprule
              &   & id & $\neg$ &   \\
            \midrule
            F & F & F  & T      & T \\
            T & F & T  & F      & T \\
            \bottomrule
        \end{tabulary}
        \begin{tabulary}{\linewidth}{@{}CC|CCCCCCCCCCCCCCCC}
            \toprule
              &   &   & $\land$ &   &   &   &   & $\overset{\not\equiv}{\neq}$ & $\lor$ & \rotatebox{90}{nor} & $\overset{\equiv}{=}$ &   & $\Leftarrow$ &   & $\Rightarrow$ & \rotatebox{90}{nand} &   \\
            \midrule
            F & F & F & F       & F & F & F & F & F                            & F      & T                   & T                     & T & T            & T & T             & T                    & T \\
            F & T & F & F       & F & F & T & T & T                            & T      & F                   & F                     & F & F            & T & T             & T                    & T \\
            T & F & F & F       & T & T & F & F & T                            & T      & F                   & F                     & T & T            & F & F             & T                    & T \\
            T & T & F & T       & F & T & F & T & F                            & T      & F                   & T                     & F & T            & F & T             & F                    & T \\
            \bottomrule
        \end{tabulary}
    \end{center}
    \subsection{Definitions}
    \begin{multicols*}{2}
        Expression is \textbf{satisfied} in state $s$ iff evaluates to \textit{true} in state s. \\
        Expression is \textbf{valid} iff it is satisfied in every state. \\
        Valid expression is called a \textbf{tautology} \\
        Expression is \textbf{satisfiable} iff there is a state in which it is satisfied. \\
        Expression is a \textbf{contradiction} iff it evaluates to \textit{false} in every state. \\
        Two expressions are \textbf{logically equivalent} iff they evaluate to same truth value in every state. \\
    \end{multicols*}

    \section{LN3}
    Propositional Calculus = Axioms + Inference Rules, Inference Rules: $\dfrac{P}{P[r:=Q]}$. A theorem of our propositional calculus is either
    1 an axiom, 2 the conclusion of an inference rule whose premises are
    theorems, or 3 a boolean expression that, using the inference rules, is proved equal to an axiom or a previously proved theorem. Heuristic: Identify applicable theorems by matching the structure of expressions or sub-expressions. The operators that appear in a boolean expression and the shape of its sub-expressions can focus the choice of theorems to be used in
    manipulating it. Principle: Structure proofs to avoid repeating the same
    sub-expression on many lines.\\

    \section{LN4}
    See: Proof techniques

    \section{LN5}
    A formal logical system, or logic, is a set of rules defined in terms
    of a set of symbols, a set of formulas constructed from the symbols,
    a set of distinguished formulas called axioms, and a set of inference rules.
    The set of formulas is called the language of the logic.
    The language is defined syntactically;
    there is no notion of meaning or semantics in a logic per se.
    A formula is a theorem of the logic, if it is one of the following:
    an axiom, can be generated from the axioms and already proved theorems
    using the inference rules.
    A proof that a formula is a theorem is an argument that shows
    how the inference rules are used to generate the formula.
    A logic is consistent if at least one of its formulas is a theorem
    and at least one is not; otherwise, the logic is inconsistent.
    Models: We give the formulas a meaning with respect to this domain,
    1 by defining which formulas are true statements about the domain,
    2 by defining which formulas are false statements about the domain.
    An interpretation assigns meaning to the: operators of a logic,
    constants of a logic and variables of a logic.
    Standard interpretation of expressions of
    (a) propositional logic For an expression P without variables,
    let eval(P) be the value of P. Let Q be any expression,
    and let s be a state that gives values to all the variables of Q.
    Define Q(s) to be a copy of Q in which all its variables are
    replaced by their corresponding values in state $s$.
    Then function f given by f (Q) = eval(Q(s)) is an interpretation for Q.\\

    \subsection{Definitions}
    Let S be a set of interpretations for a logic and F be a formula of the logic.
    F is \textbf{satisfiable} (under S) iff at least one interpretation of S maps F to true.
    F is \textbf{valid} (under S) iff every interpretation in S maps F to true.
    An \textbf{interpretation} is a model for a logic iff
    every theorem is mapped to true by the interpretation.
    A logic is \textbf{sound} iff every theorem is valid.
    A logic is \textbf{complete} iff every valid formula is a theorem.
    \textbf{Soundness} means that the theorems are true statements
    about the domain of discourse,
    \textbf{Completeness} means that every valid formula can be proved.
    A \textbf{sound and complete logic} allows exactly the valid formulas to be proved.
    A boolean expression is \textbf{satisfied} in state s
    iff it evaluates to true in state s.
    A boolean expression is \textbf{valid} iff it is satisfied in every state.
    A valid boolean expression is called a \textbf{tautology}.
    A boolean expression is \textbf{satisfiable}
    iff there is a state in which it is satisfied.
    The atomic proposition is a type of statement,
    which contains a truth value that can be true or false.\\

    \subsection{Peano Arithmetic}
    Symbols: S, o, +, ·, <, = Formulas: $\varphi$ \\
    Axioms: The axioms of PA are:
    (1) $\forall x (Sx \not = 0)$\\
    (2) $\forall x,y ((Sx = Sy) \xrightarrow{} x =y)$\\
    (3) $( \varphi[0] \land \forall x(\varphi[x] \xrightarrow{} \varphi[Sx])) \xrightarrow{} \forall x (\varphi[x]),$ for any formula $\varphi$ in PA.\\
    (4) $\forall x (x + 0) =x$ (5) $\forall x,y(x + Sy = S(x+y))$ (6) $\forall (x \cdot 0 = 0)$\\
    (7) $\forall x,y(x \cdot Sy = (x \cdot y) + x)$ \\
    Natural Deduction is a version of Propositional Logic often
    better suited for formal proofs.
    Logicians express this relationship between a theorem
    and the formulas assumed for its proof as the sequent:
    $A_0,...,A_n \vdash Q$ or $\vdash Q$,
    where L is the name of the logic with axioms $A_0,...,A_n$.
    Symbol $\vdash$ is called the "turnstile",
    and the $A_i$ are called the premises of the sequent.
    The sequent $A_0,...,A_0 \vdash$ Q is read
    as Q is provable from $A_0,...,A_n$
    (The order of the $A_i$ is immaterial.)
    The sequent $\vdash_L$ Q is read as "Q is provable in logic L" -
    i.e. using the axioms of L.\\

    \subsection{Constructive Propositional Logic}
    (1) A proof of $p \land q$ is given by presenting a proof of p and a
    proof of q (2) A proof of $p \lor q$ is given by presenting either a proof of p or a proof of q (3) A proof of $p \xrightarrow{} q$ is a procedure that permits us to transform a proof of p into a proof of q. (4) The constant false, which is a contradiction, has no proof. (5) A proof of $\neg p$ is a procedure that transforms any hypothetical proof of p into a proof of a contradiction ($p \vdash false$ i.e., false is provable from p).\\
    \textbf{Rules for Constructive Natural Deduction:}\\
    Introduction rules: \\
    ||\\
    $\land-I$: $\dfrac{\vdash P, \vdash Q}{\vdash P \land Q}$ $\lor-I_l$: $\dfrac{\vdash P}{\vdash P \lor Q}$ $\lor-I_r$: $\dfrac{\vdash Q}{\vdash P \lor Q}$\\
    $\xrightarrow{}-I$: $\dfrac{P_1,...,P_n \vdash Q}{\vdash P_1 \land ... \land P_n \rightarrow Q}$ $false-I$: (none)\\
    ||\\
    Elimination Rules:\\
    ||\\
    $\land-E \dfrac{P \land Q}{P}, \dfrac{P \land Q}{Q}$ $\lor-E \dfrac{P \land Q, P \xrightarrow{} R, Q \xrightarrow{} R}{R}$ $\xrightarrow{}-E \dfrac{P, P \xrightarrow{} Q}{Q}$\\
    $\equiv-E_l \dfrac{\vdash P \land Q}{\vdash P}, \land-E_r \dfrac{\vdash P \land Q}{Q}$ $\lor-E \dfrac{\vdash P \lor Q, P \vdash R, Q \vdash R}{\vdash R}$ $\xrightarrow{}-E: \dfrac{\vdash P, \vdash P ]\xrightarrow{} Q}{\vdash Q} F-E \dfrac{\vdash F}{\vdash P}$\\
    $P \equiv Q \text{ denotes } (P \xrightarrow{} Q) \land (Q \xrightarrow{} P) $ || $\neg P \text{ denotes } P \xrightarrow{} F$ $T \text{ denotes } \neg F$\\
    $\neg \neg p \xrightarrow{} p $ is NOT a theorem $p \xrightarrow{} \neg \neg p $ is a theorem \\ $p \lor \neg p $ is NOT a theorem $\neg \neg (p \lor \neg p)$ is a theorem.\\
    Theorem-Soundness: An inference rule is considered sound if a formula derived using it is valid whenever the premises used in the inference are theorems. Model-Soundness: An inference rule is considered sound if a
    formula derived using it is valid whenever the premises used in the inference are valid.\\

    \section{LN6}
    In a textual substitution $E[x := F], x$ and $F$ must have the same type. a notion of subtypes: for example, the natural numbers $\mathbf{N}$ are a subset of the integers $\mathbf{Z}$, so l : $\mathbf{Z}$ and l : $\mathbf{N}$ are both suitable declarations. a notion overloading: we need a notion of subtypes, as well as a notion of overloading of both constants and operators, so that the same constants and operators can be used in more than one way. a notion of polymorphism: we also need a notion of
    polymorphism; as an example function =: t $\times$ t $\xrightarrow{} \mathbf{B}$ is polymorphic because it is defined for any type t.\\
    $\Sigma^{n}_{i=1} e$ is any expression. $\Sigma(i | 1 \leq i < n: e)$ Linear notation\\
    Let $*$ be any binary operator that satisfy:\\
    Sym/Comm: $b * c = c * b$ Assoc.: $(b * c) * d = b * (c * d)$\\
    Id. u: $u * b = b = b * u$ A set of values together with an operator $*$ that satisfy the above is called an Abelian monoid.\\
    The general form of a quantification over $*$ is exemplified by
    $*(x : t1, y : t2 | R : P )$ Variables x and y are distinct. They are called the bound variables or dummies of the quantification. t1 and t2 are the types of dummies x and y If t1 and t2 are the same type, we may write
    $*(x, y : t1 | R : P )$ R, a boolean expression, is the range of the quantification R may refer to dummies x and y. If the range is omitted, as in $*(x, y : t1 |: P )$ , then the range true is meant. P, an expression, is the body of the quantification. P may refer to dummies x and y. Expression $*(x : X | R : P )$ denotes the application of operator $*$ to the values P for all x in X for which range R is true.\\
    Free and Bound occurrences in a variable:\\
    The occurrence of i in the expression i is free. Suppose an occurrence of i in expression E is free. Then that same occurrence of i is free in (E), in function application $f (..., E,...)$, and in $*(x | E : F )$ and $*(x | F : E)$ provided i is not one of the dummies in list x.\\
    Let an occurrence of i be free in an expression E. That occurrence of i is bound (to dummy i) in the expression $*x | E : F )$ and $*(x | F : E )$ if i is one of the dummies in list x. Suppose an occurrence of i is bound in expression E. Then it is also bound (to the same dummy) in (E), in function application f (..., E,...) , and in $*(x | E : F )$ and $*(x | F : E )$.\\
    \subsection{Textual Substitution}
    Provided $\neg occurs('y','x, F')$, i.e. a dummy of list y will have to
    be replaced by a fresh variable if that dummy occurs free in x or F.
    $*(y | R : P )[x := F] = *(y | R[x := F] : P[x := F])$\\
    Assume that the operator $*$ is symmetric and associative and has
    an identity u. Two additional inferences rules allow substitution of equals for equals in the range and body of a quantification (Leibniz).\\
    $\dfrac{P = Q}{*(x |E[z:= P] : S) = *(x |E[z:= Q] : S)}$\\
    $\dfrac{R \xrightarrow{} P = Q}{*(x | R : E[z:= P]) = *(x | R: E[z:= Q])}$\\
    Operation $*$ is idempotent iff $x * x = x$ for all x. Quantifiers '$\lor$', '$\land$', '$\cup$', '$\cap$' are idempotent, while '+' and '$\cdot$' are not.\\
    \section{LN7}
    A predicate-calculus formula is a boolean expression in which some boolean variables may have been replaced by: Predicates : applications of boolean functions whose arguments may be of types other than $\mathbf{B}$, Universal and existential quantification.\\

    \section{LN7a}
    $R_i(x_1,...,x_n)$ - atomic formula, n is and arity of the relational symbol $R_i$. All appearances of $R_i$ must have the same arity.\\
    $\Phi$ is a formula iff: $\Phi$ is atomic. $\Phi = \Phi _1 \land \Phi _2, \Phi = \Phi _1 \lor \Phi _2, \Phi = \Phi \neg \Phi$ where $\Phi _1$ and $\Phi _2$ are formulas. $\Phi$ = $\exists [\Phi], \Phi = \forall [\Phi]$\\
    Prenex Normal Form: All quantifiers appear in the front of
    the formula. We assume all our formulas are in prenex normal form.
    It can be proved that every formula has its equivalent prenex normal form. Free variable: not bound by any quantifier Sentence, statement: no free variables.\\
    A model (interpretation, structure) is a tuple $M = (U,P_1,...,P_k )$,
    where U is a universe over which the variables may take values, $P_i$ is a relation assigned to the symbol $R_i$. A language of a model is the set of all formulas of the model. If the formula $\Phi$ is true in a model M, we say that M is a Model of $\Phi$.\\
    $\Phi = \forall x.\forall y. R_1(x, y) \lor R_1(y, x)$
    Model $M_1: U$ - natural numbers, P1 is $\leq$ (we write $a \leq b$ instead of $\leq (a, b)$ or $P_1(a, b)) \Phi M_1 = \forall x. \forall y. x \leq y \lor y \leq x$ - the formula $\Phi M_1$ is true so $M_1$ is a model of $\Phi$. Model $M_2$: U - natural numbers, $P_1$ is < (we write $a < b$ instead of $< (a, b)$ or $ P_1(a, b))$ $\Phi M_2 = \forall x.\forall y. x < y \lor y < x$, the formula $\Phi M_2$ is false so $M_2$ is not a model of $\Phi$.\\
    Examples of English to Predicate logic:\\
    (a) The natural number 1  is the only natural number that is smaller than positive integer $p$ and divides $p$. $(\forall d | 1 < d < p : \neg (\exists v | 0 \leq v : d \cdot v = p))$
    (c) Adding two odd integers yields an even number. (Use only addition and multiplication; do not use division, mod, or predicates even.x and odd.x) $(\forall x, y : \mathbf{Z} | (\exists i, j: \mathbf{Z} |: x = 2 \cdot i + 1 \land y = 2 \cdot j +1) : (\exists k: \mathbf{Z}|: x + y = 2 \cdot k))$
    $\forall$ Everybody, $\exists$ Somebody, $\neg \exists$ Nobody.\\
    (a) Everybody loves everybody. $(\forall x :P|:(\forall y: P|: loves(x,y)))$ (b) Nobody loves everybody. $\neg (\exists x: P|: (\forall y: P|: loves(x,y)))$ (c) Somebody loves nobody. $(\exists x : P|: \neg (\exists y :P|: loves(x,y)))$
\end{multicols*}

\end{document}
